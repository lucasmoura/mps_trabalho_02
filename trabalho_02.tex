\documentclass[a4paper, 11pt]{article}
\usepackage{comment} % enables the use of multi-line comments (\ifx \fi) 
\usepackage{lipsum} %This package just generates Lorem Ipsum filler text. 
\usepackage{fullpage} % changes the margin
\usepackage[brazilian]{babel}
\usepackage[utf8]{inputenc}
\usepackage[T1]{fontenc}
\usepackage{graphicx}

\begin{document}
%Header-Make sure you update this information!!!!
\noindent
\large\textbf{Melhoria de Processo de Software / Segundo trabalho}\\
Lucas Albuquerque Medeiros de Moura \hfill 11/0015568 \\
Luciano Prestes Cavalcanti \hfill ?

\section*{Resumo do trabalho}
Este trabalho visa explicitar o processo de desenvolvimento para a evolução do
software Noosfero no contexto do Laboratório Avançado de Produção Pesquisa e
Inovação em Software (LAPPIS).

\section*{Introdução}

O Noosfero pode ser descrito como plataforma web com o foco de permitir a
customização de redes sociais, visando desde a criação de blogs a até mesmo
agenda de eventos. Tal software é atualmente usado pela Universidade de São
Paulo para gerencimento de algumas de suas disciplinas e até mesmo pelo SERPRO.

No contexto do LAPPIS, o Noosfero é usado para prover uma rede social para o
novo portal de software público. O intuito desse portal é provar uma plataforma
única de gerenciamento de softwares livre do governo, visando desde prover um
repositório d código fonte a até mesmo prover um meio para interação da
comunidade de um dado software livre. O Noosfero, dessa forma, é responsável
pela configuração desse ambiente de interação entre os membros de uma
comunidade.

Com isso em mente, o laboratório LAPPIS tem como motivação a criação de um
processo que vise tanto a manutenção de um fork do Noosfero para o contexto do
novo portal do software público e também estar alinhado e presente com o
desenvolvimento da versão principal do Noosfero. Considerando que o LAPPIS
baseia seu processo em técnicas propostas pela metodologia ágil, todo o processo
de desenvolvimento é encapsulado em uma sprint de trabalho, que normalmente tem
duração de duas semanas.

Sendo assim, durante uma sprint, o processo de desenvolvimento usado segue as
seguintes atividades:

\begin{itemize}
    \item Identificar funcionalidades
    \item Desenvolver funcionalidades
    \item Analisar resultados
\end{itemize}

O objetivo desse processo é então, basicamente, a criação de novas funcionalides
para o Noosfero no contexto do software público de forma dinâmica e de rápida
assimilação pelo time.

Com o processo minimamente explicado, pode-se então explicar como esse documento
será estruturado. Esse documento é composto por duas sessões, Desenvolvimento e Considerações
finais. A sessão de Desenvolvimento é usado para o detalhamento do objetivo e das atividades de
cada etapa do processo, além de explicar como tais atividades são detalhadas e
quais ferramentas são usadas para o acompanhamento do processo.


\section*{Desenvolvimento}

Considerando as etapas levantadas na seção anterior, é necessário descrever
melhor o motivo e as atividades relacionadas a cada uma delas. Esta seção visa
então realizar tal papel, além de mostrar em linhas gerais quais as ferramentas
e papéis usados para o gerenciamento do processo como um todo.

\subsection*{Identificar Funcionalidades}

Esta etapa do processo é usada para selecionar quais tarefas precisam ser
realizadas dentro da sprint, levando em consideração a prioridade de cada uma
das tarefas e a composição do time. Para realizar tal etapa, a tabela à seguir
descreve as atividades necessárias para a execução dessa etapa:

\begin{table}[h]
    \resizebox{\textwidth}{!}{\begin{tabular}{|l|l|}
		\hline
        Atividade          & Função \\ \hline
        Definir tarefas    & \parbox{10cm}{Definir quais são as atividades que precisam ser realizadas para a sprint, ou até mesmo, em algum momento do projeto.\\} \\ \hline
        Priorizar tarefas  & \parbox{10cm}{Com as tarefas definidas, deve-se priorizar quais são as principais tarefas que devem ser desempanhadas na duração da sprint. Essa tarefa deve ser realizada em conjunto com todo o time relacionado ao Noosfero.\\}\\ \hline
        Definir duplas     & \parbox{10cm}{Com as tarefas priorizadas, deve-se definir quais duplas devem ser responsáveis pelo desenvolvimento de cada tarefa.\\}\\ \hline
        Documentar tarefas & \parbox{10cm}{Com as duplas definidas, cada dupla deve documentar suas tarefas e documentar qualquer subtarefa relacionada a mesma. Além disso, a dupla também deve ser responsável por definir o critério de aceitação de uma dada tarefa.\\} \\
		\hline
	\end{tabular}}
    \caption{Atividades relacionadas a etapa de Identificar funcionalidades}
    \label{tab:atividades_identificar}
\end{table}

\subsection*{Desenvolver funcionalidade}

Esta etapa do processo visa realizar uma tarefa proposta.

\section*{Optimum Solution}
\lipsum[4]
% to comment sections out, use the command \ifx and \fi. Use this technique when
% writing your pre lab. For example, to comment something out I would do:
%  \ifx
%   \begin{itemize}
%       \item item1
%       \item item2
%   \end{itemize}   
%  \fi

\section*{Construction/Implementation}
\lipsum[5]

\section*{Analysis \& Testing}
\lipsum[6]

\section*{Final Evaluation}
\lipsum[7]

\section*{Attachments}
%Make sure to change these
Lab Notes, HelloWorld.ic, FooBar.ic
%\fi %comment me out

\begin{thebibliography}{9}
\bibitem{Robotics} Fred G. Martin \emph{Robotics Explorations: A Hands-On
Introduction to Engineering}. New Jersey: Prentice Hall.
\bibitem{Flueck}  Flueck, Alexander J. 2005. \emph{ECE 100}[online]. Chicago:
Illinois Institute of Technology, Electrical and Computer Engineering
Department, 2005 [cited 30
August 2005]. Available from World Wide Web:
(http://www.ece.iit.edu/~flueck/ece100).
\end{thebibliography}

\end{document}

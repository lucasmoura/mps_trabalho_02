\documentclass[a4paper, 11pt]{article}
\usepackage{comment} % enables the use of multi-line comments (\ifx \fi) 
\usepackage{lipsum} %This package just generates Lorem Ipsum filler text. 
\usepackage{fullpage} % changes the margin
\usepackage[brazilian]{babel}
\usepackage[utf8]{inputenc}
\usepackage[T1]{fontenc}
\usepackage{graphicx}

\begin{document}
%Header-Make sure you update this information!!!!
\noindent
\large\textbf{Melhoria de Processo de Software / Segundo trabalho}\\
Lucas Albuquerque Medeiros de Moura \hfill 11/0015568 \\
Luciano Prestes Cavalcanti \hfill ?

\section*{Resumo do trabalho}
Este trabalho visa explicitar o processo de desenvolvimento para a evolução do
software Noosfero no contexto do Laboratório Avançado de Produção Pesquisa e
Inovação em Software (LAPPIS).

\section*{Introdução}

O Noosfero pode ser descrito como plataforma web com o foco de permitir a
customização de redes sociais, visando desde a criação de blogs a até mesmo
agenda de eventos. Tal software é atualmente usado pela Universidade de São
Paulo para gerencimento de algumas de suas disciplinas e até mesmo pelo SERPRO.

No contexto do LAPPIS, o Noosfero é usado para prover uma rede social para o
novo portal de software público. O intuito desse portal é provar uma plataforma
única de gerenciamento de softwares livre do governo, visando desde prover um
repositório d código fonte a até mesmo prover um meio para interação da
comunidade de um dado software livre. O Noosfero, dessa forma, é responsável
pela configuração desse ambiente de interação entre os membros de uma
comunidade.

Com isso em mente, o laboratório LAPPIS tem como motivação a criação de um
processo que vise tanto a manutenção de um fork do Noosfero para o contexto do
novo portal do software público e também estar alinhado e presente com o
desenvolvimento da versão principal do Noosfero. Considerando que o LAPPIS
baseia seu processo em técnicas propostas pela metodologia ágil, todo o processo
de desenvolvimento é encapsulado em uma sprint de trabalho, que normalmente tem
duração de duas semanas.

Sendo assim, durante uma sprint, o processo de desenvolvimento usado segue as
seguintes atividades:

\begin{itemize}
    \item Identificar funcionalidades
    \item Desenvolver funcionalidades
    \item Analisar resultados
\end{itemize}

O objetivo desse processo é então, basicamente, a criação de novas funcionalides
para o Noosfero no contexto do software público de forma dinâmica e de rápida
assimilação pelo time.

Com o processo minimamente explicado, pode-se então explicar como esse documento
será estruturado. Esse documento é composto por duas sessões, Desenvolvimento e Considerações
finais. A sessão de Desenvolvimento é usado para o detalhamento do objetivo e das atividades de
cada etapa do processo, além de explicar como tais atividades são detalhadas e
quais ferramentas são usadas para o acompanhamento do processo.


\section*{Desenvolvimento}

Considerando as etapas levantadas na seção anterior, é necessário descrever
melhor o motivo e as atividades relacionadas a cada uma delas. Esta seção visa
então realizar tal papel, além de mostrar em linhas gerais quais as ferramentas
e papéis usados para o gerenciamento do processo como um todo.

\subsection*{Identificar Funcionalidades}

Esta etapa do processo é usada para selecionar quais tarefas precisam ser
realizadas dentro da sprint, levando em consideração a prioridade de cada uma
das tarefas e a composição do time. Para realizar tal etapa, a tabela à seguir
descreve as atividades necessárias para a execução dessa etapa:

\begin{table}[h]
    \resizebox{\textwidth}{!}{\begin{tabular}{|l|l|}
		\hline
        Atividade          & Função \\ \hline
        Definir tarefas    & \parbox{10cm}{Definir quais são as atividades que precisam ser realizadas para a sprint, ou até mesmo, em algum momento do projeto.\\} \\ \hline
        Priorizar tarefas  & \parbox{10cm}{Com as tarefas definidas, deve-se priorizar quais são as principais tarefas que devem ser desempanhadas na duração da sprint. Essa tarefa deve ser realizada em conjunto com todo o time relacionado ao Noosfero.\\}\\ \hline
        Definir duplas     & \parbox{10cm}{Com as tarefas priorizadas, deve-se definir quais duplas devem ser responsáveis pelo desenvolvimento de cada tarefa.\\}\\ \hline
        Documentar tarefas & \parbox{10cm}{Com as duplas definidas, cada dupla deve documentar suas tarefas e documentar qualquer subtarefa relacionada a mesma. Além disso, a dupla também deve ser responsável por definir o critério de aceitação de uma dada tarefa.\\} \\
		\hline
	\end{tabular}}
    \caption{Atividades relacionadas a etapa de Identificar funcionalidades}
    \label{tab:atividades_identificar}
\end{table}

\subsection*{Desenvolver funcionalidade}

Esta etapa do processo visa realizar uma tarefa proposta. Apesar da abordagem de
como abordar uma tarefa seja responsabilidade da dupla destinada a realizar tal
tarefa, existem uma categorização informal de como tratar alguns tipos de
tarefas:

\begin{itemize}
    \item \textbf{Funcionalidades para o SPB:} Quando uma funcionalidade deve
        ser desenvolvida só para o contexto do SPB, a dupla responsável deve
        fazer as alterações somente no fork do Noosfero para o SPB. Para este
        tipo de tarefa, não é explicitamente necessário a interação da dupla com
        os mantenedores do Noosfero, mas obviamente, tal interação não é
        restrita.
    \item \textbf{Funcionalidades para o Noosfero:} Quando uma funcionalidade
        proposta pode empactar usuários do Noosfero que não estão só
        relacionados ao contexto do SPB, tal funcionalidade ou modificação deve
        ser feita no repositório principal do Noosfero. Para isso, é necessário
        que a dupla se organize para interagir diretamente com os mantenedores
        do Noosfero, pois tal interação ocorre normalmente de forma não
        presencial.
    \item \textbf{Correção de bugs:} Neste tipo de tarefa, normalmente tenta-se
        alocar os responsáveis iniciais pela tarefa que gerou um bug. Caso isso
        não seja possível, deve-se minimanete tentar o contato com a dupla
        original e tentar nortear o que pode estar causando o bug, com o intuito
        de poupar esforço inicial de encontrar onde está o bug, e sim focar em
        consertar o problema.
\end{itemize}

Apesar desses três tipos de tarefas principais, existem também tarefas
relacionadas a documentação e tradução do Noosfero. Entretanto, tais tarefas
seguem o mesmo fluxo de atividades que será detalhada a seguir, e não carecem de
nenhuma explicação mais detalhada.

Considerando os tipos de tarefas realizadas pela equipe, tornou-se necessário
a descrição das seguintes atividades para esta etapa do processo:


\begin{table}[h]
    \resizebox{\textwidth}{!}{\begin{tabular}{|l|l|}
		\hline
        Atividade          & Função \\ \hline
        Desenvolver solução    & \parbox{10cm}{Esta atividade está relacionada
        ao trabalho efetivo para implementar uma solução para uma dada tarefa da
        dupla.\\} \\ \hline
        Atualizar tarefa  & \parbox{10cm}{Conforme o andamento da tarefa, a
        dupla deve sempre atualizar como está o andamento das atividades, como
        por exemplo, o que já está terminado e o que está sendo feito.\\}\\ \hline
        Solicitar revisão     & \parbox{10cm}{Com a tarefa totalmente concluída,
        a dupla deve pedir para que outros membros do time ou, dependendo do
        tipo de tarefa sendo desenvolvida, um mantenedor do Noosfero, que revise
        a solução desenvolvida. A dupla pode apontar que ela acha melhor para
        revisar tal tarefa, mas tal apontamento não é fixo e pode ser alterado
        pelas necessidades do time. Caso o revisor da tarefa encontre algum
        problema na solução encontrada, a dupla deve voltar a atividde de
        desenvolver solução e corrigir os problemas apontados.\\}\\ \hline
        Relatar status da tarefa & \parbox{10cm}{Durante a execução da sprint, a
        dupla deve relatar o andamento de sua tarefa. Tal relato não é feito só
        para o time do Noosfero propriamente dito, mas sim para para todo o time
        do SPB, visando que todos os membros estejam alinhados quanto ao
        objetivo da sprint e dos projetos ao seu redor.\\} \\
		\hline
	\end{tabular}}
    \caption{Atividades relacionadas a etapa de Desenvolver funcionalidades}
    \label{tab:atividades_desenvolver}
\end{table}

\subsection*{Analisar resultados}

Esta etapa do projeto acontece quando a sprint chega ao seu término. Dessa
forma, esta etapa visa entender o que foi realizado na sprint, seu alguma tarefa
na sprint não foi concluída e entender o que pode ser melhorado na próxima
sprint.

Dessa forma, esta etapa possui as seguintes atividades:

\begin{table}[h]
    \resizebox{\textwidth}{!}{\begin{tabular}{|l|l|}
		\hline
        Atividade          & Função \\ \hline
        Realizar reunião de fim da sprint& \parbox{10cm}{Esta atividade acontece
        ao término de uma sprint, onde ambas as tarefas encerradas e pendentes
        são reportadas para todo o time. Além disso, nessa reunião também é
        levantado os problemas encontrados durante a execução da sprint e o que
        o time pode fazer para mitigar tais dificuldades no próximo ciclo de
        desenvolvimento.\\} \\ \hline
        Documentar sprint& \parbox{10cm}{Após a reunião de término, o time deve
        documentar o que foi realizado na sprint. Isso é tudo realizado na wiki
        do projeto, que dita o que foi feito no projeto e o que ainda precisa
        ser feito.\\}\\ \hline
        Atualizer tarefas& \parbox{10cm}{Dependendo da tarefa, a sua
        documentação pode carecer de atualização ao término de uma sprint. Por
        exemplo, caso o seu motivo de existir tenha sido alterado ou até mesmo
        se seus critérios de aceitação forem alterados. Dessa forma, a dupla
        responsável pela tarefa deve atualizar tais informações.\\}\\ \hline
	\end{tabular}}
    \caption{Atividades relacionadas a etapa de Desenvolver funcionalidades}
    \label{tab:atividades_desenvolver}
\end{table}

\subsection*{Ferramentas}

Para gerenciar o processo criado para o projeto SPB, as seguintes ferramentas
são utilizadas:


\begin{table}[h]
    \resizebox{\textwidth}{!}{\begin{tabular}{|l|l|}
		\hline
        Ferramenta          & Função \\ \hline
        Gitlab & \parbox{10cm}{Ferramenta usada como repositório de código
        fonte para o Noosfero, além de gerenciar as tarefas que são
        desenvolvidas e também a wiki do projeto, suportando o acompanhamento
        das sprints. \\} \\ \hline
        IRC & \parbox{10cm}{Usado para comunicação entre membros do time e com
        os mantenedores do Noosfero.\\}\\ \hline
        Lista de email& \parbox{10cm}{Usado para comunicação mais detalhada
        sobre andamento da sprint ou problemas no decorrer da mesma.\\}\\ \hline
	\end{tabular}}
    \caption{Atividades relacionadas a etapa de Desenvolver funcionalidades}
    \label{tab:atividades_desenvolver}
\end{table}

\subsection*{Papéis}

No contexto do projeto SPB, existem quatro papéis distintos no processo:

\begin{itemize}
    \item \textbf{Desenvolvedor:} Responsável pelo desenvolvimento de
        funcionalidade e correção de problemas.
    \item \textbf{Couch:} Responsável por ajudar o time a se organizar e se um
        ponto central quanto a questionamentos sobre andamentos de tarefa.
    \item \textbf{Meta couch:} Responsável por fazer a ponte entre os diferentes
        time do projeto. Muitas vezes, é alocado em tarefas de integração entre
        as ferramentas.
    \item \textbf{Sênior:} Desenvolvedor mais experiente no projeto. Normalmente
        recebe as funcionalidades mais complexas e ajuda o time quanto a dúvidas
        técnicas.
\end{itemize}

Vale ressaltar, que mesmo com essa diferença de papéis, todos os membros do
projeto são responsáveis pelo desenvolvimento, apenas papéis tem abordagens
diferentes quanto a isso.

\section*{Final Evaluation}
\lipsum[7]

\section*{Attachments}
%Make sure to change these
Lab Notes, HelloWorld.ic, FooBar.ic
%\fi %comment me out

\begin{thebibliography}{9}
\bibitem{Robotics} Fred G. Martin \emph{Robotics Explorations: A Hands-On
Introduction to Engineering}. New Jersey: Prentice Hall.
\bibitem{Flueck}  Flueck, Alexander J. 2005. \emph{ECE 100}[online]. Chicago:
Illinois Institute of Technology, Electrical and Computer Engineering
Department, 2005 [cited 30
August 2005]. Available from World Wide Web:
(http://www.ece.iit.edu/~flueck/ece100).
\end{thebibliography}

\end{document}
